%%%%%%%%%%%%%%%%%%%%%%%%%%%%%%%%%%%%%%%%%
% Programming/Coding Assignment
% LaTeX Template
%
% This template has been downloaded from:
% http://www.latextemplates.com
%
% Original author:
% Ted Pavlic (http://www.tedpavlic.com)
%
% Note:
% The \lipsum[#] commands throughout this template generate dummy text
% to fill the template out. These commands should all be removed when 
% writing assignment content.
%
% This template uses a Perl script as an example snippet of code, most other
% languages are also usable. Configure them in the "CODE INCLUSION 
% CONFIGURATION" section.
%
%%%%%%%%%%%%%%%%%%%%%%%%%%%%%%%%%%%%%%%%%

%----------------------------------------------------------------------------------------
%	PACKAGES AND OTHER DOCUMENT CONFIGURATIONS
%----------------------------------------------------------------------------------------

%\documentclass{article}
\documentclass[12pt]{article}
\usepackage{fancyhdr} % Required for custom headers
\usepackage{lastpage} % Required to determine the last page for the footer
\usepackage{extramarks} % Required for headers and footers
\usepackage[usenames,dvipsnames]{color} % Required for custom colors
\usepackage{graphicx} % Required to insert images
\usepackage{subcaption}
\usepackage{listings} % Required for insertion of code
\usepackage{courier} % Required for the courier font
\usepackage{amsmath}
\usepackage{framed}

% Margins
\topmargin=-0.45in
\evensidemargin=0in
\oddsidemargin=0in
\textwidth=6.5in
\textheight=9.0in
\headsep=0.25in

\linespread{1.1} % Line spacing

% Set up the header and footer
\pagestyle{fancy}
\lhead{\hmwkAuthorName} % Top left header
\chead{\hmwkClass\ (\hmwkClassTime): \hmwkTitle} % Top center head
%\rhead{\firstxmark} % Top right header
\lfoot{\lastxmark} % Bottom left footer
\cfoot{} % Bottom center footer
\rfoot{Page\ \thepage\ of\ \protect\pageref{LastPage}} % Bottom right footer
\renewcommand\headrulewidth{0.4pt} % Size of the header rule
\renewcommand\footrulewidth{0.4pt} % Size of the footer rule

\setlength\parindent{0pt} % Removes all indentation from paragraphs

%----------------------------------------------------------------------------------------
%	CODE INCLUSION CONFIGURATION
%----------------------------------------------------------------------------------------

\definecolor{mygreen}{rgb}{0,0.6,0}
\definecolor{mygray}{rgb}{0.5,0.5,0.5}
\definecolor{mymauve}{rgb}{0.58,0,0.82}

\lstset{ %
  backgroundcolor=\color{white},   % choose the background color
  basicstyle=\footnotesize,        % size of fonts used for the code
  breaklines=true,                 % automatic line breaking only at whitespace
  captionpos=b,                    % sets the caption-position to bottom
  commentstyle=\color{mygreen},    % comment style
  escapeinside={\%*}{*)},          % if you want to add LaTeX within your code
  keywordstyle=\color{blue},       % keyword style
  stringstyle=\color{mymauve},     % string literal style
}

%----------------------------------------------------------------------------------------
%	DOCUMENT STRUCTURE COMMANDS
%	Skip this unless you know what you're doing
%----------------------------------------------------------------------------------------

% Header and footer for when a page split occurs within a problem environment
\newcommand{\enterProblemHeader}[1]{
%\nobreak\extramarks{#1}{#1 continued on next page\ldots}\nobreak
%\nobreak\extramarks{#1 (continued)}{#1 continued on next page\ldots}\nobreak
}

% Header and footer for when a page split occurs between problem environments
\newcommand{\exitProblemHeader}[1]{
%\nobreak\extramarks{#1 (continued)}{#1 continued on next page\ldots}\nobreak
%\nobreak\extramarks{#1}{}\nobreak
}

\setcounter{secnumdepth}{0} % Removes default section numbers
\newcounter{homeworkProblemCounter} % Creates a counter to keep track of the number of problems
\setcounter{homeworkProblemCounter}{0}

\newcommand{\homeworkProblemName}{}
\newenvironment{homeworkProblem}[1][Problem \arabic{homeworkProblemCounter}]{ % Makes a new environment called homeworkProblem which takes 1 argument (custom name) but the default is "Problem #"
\stepcounter{homeworkProblemCounter} % Increase counter for number of problems
\renewcommand{\homeworkProblemName}{#1} % Assign \homeworkProblemName the name of the problem
\section{\homeworkProblemName} % Make a section in the document with the custom problem count
\enterProblemHeader{\homeworkProblemName} % Header and footer within the environment
}{
\exitProblemHeader{\homeworkProblemName} % Header and footer after the environment
}

\newcommand{\problemAnswer}[1]{ % Defines the problem answer command with the content as the only argument
\noindent\framebox[\columnwidth][c]{\begin{minipage}{0.98\columnwidth}#1\end{minipage}} % Makes the box around the problem answer and puts the content inside
}

\newcommand{\homeworkSectionName}{}
\newenvironment{homeworkSection}[1]{ % New environment for sections within homework problems, takes 1 argument - the name of the section
\renewcommand{\homeworkSectionName}{#1} % Assign \homeworkSectionName to the name of the section from the environment argument
\subsection{\homeworkSectionName} % Make a subsection with the custom name of the subsection
\enterProblemHeader{\homeworkProblemName\ [\homeworkSectionName]} % Header and footer within the environment
}{
\enterProblemHeader{\homeworkProblemName} % Header and footer after the environment
}

%----------------------------------------------------------------------------------------
%	NAME AND CLASS SECTION
%----------------------------------------------------------------------------------------

\newcommand{\hmwkTitle}{Written Homework 2} % Assignment title
\newcommand{\hmwkDueDate}{Monday, Feb 11, 2019} % Due date
\newcommand{\hmwkClass}{CSC421} % Course/class
\newcommand{\hmwkClassTime}{LEC 5101} % Class/lecture time
\newcommand{\hmwkAuthorName}{Zhongtian Ouyang} % Your name
\newcommand{\hmwkAuthorID}{1002341012} % Your ID


%----------------------------------------------------------------------------------------
%	TITLE PAGE
%----------------------------------------------------------------------------------------

\title{
\vspace{2in}
\textmd{\textbf{\hmwkClass:\ \hmwkTitle}}\\
\normalsize\vspace{0.1in}\small{Due\ on\ \hmwkDueDate}\\
\vspace{0.1in}
\vspace{3in}
}

\author{\textbf{\hmwkAuthorName}\\ \textbf{\hmwkAuthorID}}

\date{} % Insert date here if you want it to appear below your name

%----------------------------------------------------------------------------------------\
\begin{document}

\maketitle
\clearpage

%----------------------------------------------------------------------------------------
%	Common Tools
%----------------------------------------------------------------------------------------
%\begin{framed}
%\begin{lstlisting}[language=matlab]
%\end{lstlisting}
%\end{framed}

% \begin{bmatrix}
%0.5 & 0.6 \\ 
%0.7 & 0.8
%\end{bmatrix}

%\begin{figure}[h!]
%\centering
%\includegraphics[width=0.6\linewidth]{q10a.png}
%\label{fig:q10a}
%\end{figure}\\

%\begin{figure*}[!ht]
%\begin{subfigure}{.5\textwidth}
% \centering
%  \includegraphics[width=.5\linewidth]{p4_1.JPG}
%  \caption{Full set}
%  \label{fig:sfig1}
%\end{subfigure}
%\begin{subfigure}{.5\textwidth}
% \centering
%  \includegraphics[width=.5\linewidth]{P4_2.JPG}
%  \caption{Two each}
%  \label{fig:sfig2}
%\end{subfigure}%
%\caption{Part4 (a)}
%\label{fig:p4a}
%\end{figure*}

%\sum_{n=1}^{\infty} 2^{-n} = 1
%\prod_{i=a}^{b} f(i)

%\nabla %gradient symbol
%----------------------------------------------------------------------------------------
%	PROBLEM 1
%----------------------------------------------------------------------------------------

% To have just one problem per page, simply put a \clearpage after each problem
\begin{homeworkProblem}

\noindent \textit{Adam}\\
(a)\\
$(\alpha_A, \beta_1, \beta_2, \epsilon_A) = (\alpha_R, 0, \gamma, \epsilon_R)$\\
Since Adam is a combination of RMSprop and Momentum. If we don't do the momentum part, it will be identical to RMSprop.\\
By setting the hyperparameters of Adam as above, \\
$m_t = \beta_1m_{t-1} + (1-\beta_1)g_t = 0m_{t-1} + (1-0)g_t = g_t$, \\
$v_t = \beta_2 v_{t-1} + (1-\beta_2)g_t^2 = \gamma v_{t-1}+ (1-\gamma)g_t^2$\\
$\theta_t = \theta_{t-1} - \alpha_R g_t/(\sqrt{(v_t)} + \epsilon_R) = \theta_{t-1} - \alpha_A m_t/(\sqrt{v_t} + \epsilon_A) = \theta_{t-1} - \alpha_R g_t/(\sqrt{v_t} + \epsilon_R) $\\
which is identical to RMSprop\\

(b)\\
$(\alpha_A, \beta_1, \beta_2, \epsilon_A) = (\alpha_S, \mu, 1, 1)$\\
We can make Adam and SGD with momentum equivalent by satisfying the following conditions: $m_t = -p_t$, $(\sqrt{v_t} + \epsilon_R) = 1$ and $\alpha_A = \alpha_S$\\

If $v_t = 0$, $\epsilon_A = 1$, $\sqrt{v_t} + \epsilon_A = 1$. We can achieve this by setting $\beta_2$ to 1. Since $v_0  = 0$, $v_1, v_2 ... v_t = v_0 = 0$\\

We can prove by induction that $m_t = -p_t$ when $\beta_1 =  \mu$. \\
$m_0 = 0 = -p_0$,\\
$m_1 = \beta_1 m_0 + (1 - \beta_1)g_1 = (1 - \beta_1)g_1 = (1 - \mu)\nabla J(\theta_0)$, $p_1 = \mu p_0 - (1-\mu)\nabla J(\theta_0) = - (1-\mu)\nabla J(\theta_0)$. $m_1 = -p_1$\\
Assume $m_{t-1} = -p_{t-1}$, $m_t = \beta_1 m_{t-1} + (1 - \beta_1)g_t = -\mu p_{t-1} +  (1 - \mu)\nabla J(\theta_{t-1}) = -(\mu p_{t-1} -  (1 - \mu)\nabla J(\theta_{t-1}))$, $p_t = \mu p_{t-1} -  (1 - \mu)\nabla J(\theta_{t-1})$. $m_t = -p_t$\\

By setting the hyperparameters of Adam as above, $\theta_t = \theta_{t-1} - \alpha_A m_t/ (\sqrt{v_t} + \epsilon_A) = \theta_{t-1} + \alpha_S p_t/1 = \theta_{t-1} + \alpha_S p_t$, which is identical to SGD with momentum.\\
\clearpage
(c)\\
$\tilde L(y,t) = CL(y,t)$, $\nabla\tilde{J}(\theta_0) = C \nabla J(\theta_0)$, $\tilde g_t = Cg_t$\\

Prove $\tilde m_t = C m_t$ for $t \geq 1$:\\
Base Case:\\
$\tilde m_1 = \beta_1 \tilde m_0 + (1-\beta_1)\tilde g_1 = C(1-\beta_1)g_1$, $m_1 = \beta_1 m_0 + (1-\beta_1) g_1 = (1-\beta_1)g_1$. \\
$\tilde  m_1 = C m_1$.\\
Inductive Steps: Assume $\tilde m_{t-1} = C m_{t-1}$\\
$\tilde m_t = \beta_1 \tilde m_{t-1} + (1-\beta_1)\tilde g_t = C\beta_1m_{t-1} + C(1-\beta_1)g_t = C(\beta_1m_{t-1} + (1-\beta_1)g_t)$, \\
$m_t = \beta_1m_{t-1} + (1-\beta_1)g_t$. \\
$\tilde  m_t= C m_t$.\\

Prove $\tilde v_t = C^2 v_t$ for $t \geq 1$:\\
Base Case:\\
$\tilde v_1 = \beta_2 \tilde v_0 + (1-\beta_2)\tilde g_1^2 = C^2(1-\beta_2)g_1^2$, $v_1 = \beta_2 v_0 + (1-\beta_2) g_1^2 = (1-\beta_2)g_1^2$.\\
$\tilde  v_1 = C^2 v_1$.\\
Inductive Steps: Assume $\tilde v_{t-1} = C^2 v_{t-1}$\\
$\tilde v_t = \beta_2 \tilde v_{t-1} + (1-\beta_2)\tilde g_t^2 =C^2 \beta_2v_{t-1} + C^2(1-\beta_2)g_t^2 = C^2(\beta_2 v_{t-1} +(1-\beta_2)g_t^2)$, \\
$v_t = \beta_2 v_{t-1} +(1-\beta_2)g_t^2$. \\
$\tilde  v_t = C^2 v_t$.\\

Prove $\tilde \theta_t =\theta_t$ for $t \geq 0$:
Base Case:
$\tilde \theta_0 =\theta_0$ because they initialize to the same value
Inductive Steps: Assume $\tilde \theta_{t-1} =  \theta_{t-1}$\\
$\tilde \theta_t = \tilde \theta_{t-1} - \alpha_A\tilde m_t/(\sqrt{\tilde v_t} + \epsilon_A) = \theta_{t-1} - \alpha_A C m_t/(\sqrt{C^2 v_t} + 0) =\theta_{t-1} - \alpha_A C m_t/ C\sqrt{v_t} = \theta_{t-1} - \alpha_A m_t/\sqrt{v_t}$, \\
$\theta_t = \theta_{t-1} - \alpha_A m_t/(\sqrt{v_t} + \epsilon_A) = \theta_{t-1} - \alpha_A m_t/(\sqrt{v_t} + 0)  = \theta_{t-1} - \alpha_A m_t/\sqrt{v_t}$\\
$\tilde \theta_t = \theta_t$
\end{homeworkProblem}
%----------------------------------------------------------------------------------------

\end{document}
