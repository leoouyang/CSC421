%%%%%%%%%%%%%%%%%%%%%%%%%%%%%%%%%%%%%%%%%
% Programming/Coding Assignment
% LaTeX Template
%
% This template has been downloaded from:
% http://www.latextemplates.com
%
% Original author:
% Ted Pavlic (http://www.tedpavlic.com)
%
% Note:
% The \lipsum[#] commands throughout this template generate dummy text
% to fill the template out. These commands should all be removed when 
% writing assignment content.
%
% This template uses a Perl script as an example snippet of code, most other
% languages are also usable. Configure them in the "CODE INCLUSION 
% CONFIGURATION" section.
%
%%%%%%%%%%%%%%%%%%%%%%%%%%%%%%%%%%%%%%%%%

%----------------------------------------------------------------------------------------
%	PACKAGES AND OTHER DOCUMENT CONFIGURATIONS
%----------------------------------------------------------------------------------------

%\documentclass{article}
\documentclass[11pt]{article}
\usepackage{fancyhdr} % Required for custom headers
\usepackage{lastpage} % Required to determine the last page for the footer
\usepackage{extramarks} % Required for headers and footers
\usepackage[usenames,dvipsnames]{color} % Required for custom colors
\usepackage{graphicx} % Required to insert images
\usepackage{subcaption}
\usepackage{listings} % Required for insertion of code
\usepackage{courier} % Required for the courier font
\usepackage{amsmath}
\usepackage{framed}

% Margins
\topmargin=-0.45in
\evensidemargin=0in
\oddsidemargin=0in
\textwidth=6.5in
\textheight=9.0in
\headsep=0.25in

\linespread{1.1} % Line spacing

% Set up the header and footer
\pagestyle{fancy}
\lhead{\hmwkAuthorName} % Top left header
\chead{\hmwkClass\ (\hmwkClassTime): \hmwkTitle} % Top center head
%\rhead{\firstxmark} % Top right header
\lfoot{\lastxmark} % Bottom left footer
\cfoot{} % Bottom center footer
\rfoot{Page\ \thepage\ of\ \protect\pageref{LastPage}} % Bottom right footer
\renewcommand\headrulewidth{0.4pt} % Size of the header rule
\renewcommand\footrulewidth{0.4pt} % Size of the footer rule

\setlength\parindent{0pt} % Removes all indentation from paragraphs

%----------------------------------------------------------------------------------------
%	CODE INCLUSION CONFIGURATION
%----------------------------------------------------------------------------------------

\definecolor{mygreen}{rgb}{0,0.6,0}
\definecolor{mygray}{rgb}{0.5,0.5,0.5}
\definecolor{mymauve}{rgb}{0.58,0,0.82}

\lstset{ %
  backgroundcolor=\color{white},   % choose the background color
  basicstyle=\footnotesize,        % size of fonts used for the code
  breaklines=true,                 % automatic line breaking only at whitespace
  captionpos=b,                    % sets the caption-position to bottom
  commentstyle=\color{mygreen},    % comment style
  escapeinside={\%*}{*)},          % if you want to add LaTeX within your code
  keywordstyle=\color{blue},       % keyword style
  stringstyle=\color{mymauve},     % string literal style
}

%----------------------------------------------------------------------------------------
%	DOCUMENT STRUCTURE COMMANDS
%	Skip this unless you know what you're doing
%----------------------------------------------------------------------------------------

% Header and footer for when a page split occurs within a problem environment
\newcommand{\enterProblemHeader}[1]{
%\nobreak\extramarks{#1}{#1 continued on next page\ldots}\nobreak
%\nobreak\extramarks{#1 (continued)}{#1 continued on next page\ldots}\nobreak
}

% Header and footer for when a page split occurs between problem environments
\newcommand{\exitProblemHeader}[1]{
%\nobreak\extramarks{#1 (continued)}{#1 continued on next page\ldots}\nobreak
%\nobreak\extramarks{#1}{}\nobreak
}

\setcounter{secnumdepth}{0} % Removes default section numbers
\newcounter{homeworkProblemCounter} % Creates a counter to keep track of the number of problems
\setcounter{homeworkProblemCounter}{0}

\newcommand{\homeworkProblemName}{}
\newenvironment{homeworkProblem}[1][Problem \arabic{homeworkProblemCounter}]{ % Makes a new environment called homeworkProblem which takes 1 argument (custom name) but the default is "Problem #"
\stepcounter{homeworkProblemCounter} % Increase counter for number of problems
\renewcommand{\homeworkProblemName}{#1} % Assign \homeworkProblemName the name of the problem
\section{\homeworkProblemName} % Make a section in the document with the custom problem count
\enterProblemHeader{\homeworkProblemName} % Header and footer within the environment
}{
\exitProblemHeader{\homeworkProblemName} % Header and footer after the environment
}

\newcommand{\problemAnswer}[1]{ % Defines the problem answer command with the content as the only argument
\noindent\framebox[\columnwidth][c]{\begin{minipage}{0.98\columnwidth}#1\end{minipage}} % Makes the box around the problem answer and puts the content inside
}

\newcommand{\homeworkSectionName}{}
\newenvironment{homeworkSection}[1]{ % New environment for sections within homework problems, takes 1 argument - the name of the section
\renewcommand{\homeworkSectionName}{#1} % Assign \homeworkSectionName to the name of the section from the environment argument
\subsection{\homeworkSectionName} % Make a subsection with the custom name of the subsection
\enterProblemHeader{\homeworkProblemName\ [\homeworkSectionName]} % Header and footer within the environment
}{
\enterProblemHeader{\homeworkProblemName} % Header and footer after the environment
}

%----------------------------------------------------------------------------------------
%	NAME AND CLASS SECTION
%----------------------------------------------------------------------------------------

\newcommand{\hmwkTitle}{Written Homework 5} % Assignment title
\newcommand{\hmwkDueDate}{Thrusday, April 4, 2019} % Due date
\newcommand{\hmwkClass}{CSC421} % Course/class
\newcommand{\hmwkClassTime}{LEC 5101} % Class/lecture time
\newcommand{\hmwkAuthorName}{Zhongtian Ouyang} % Your name
\newcommand{\hmwkAuthorID}{1002341012} % Your ID


%----------------------------------------------------------------------------------------
%	TITLE PAGE
%----------------------------------------------------------------------------------------

\title{
\vspace{2in}
\textmd{\textbf{\hmwkClass:\ \hmwkTitle}}\\
\normalsize\vspace{0.1in}\small{Due\ on\ \hmwkDueDate}\\
\vspace{0.1in}
\vspace{3in}
}

\author{\textbf{\hmwkAuthorName}\\ \textbf{\hmwkAuthorID}}

\date{} % Insert date here if you want it to appear below your name

%----------------------------------------------------------------------------------------\
\begin{document}

\maketitle
\clearpage

%----------------------------------------------------------------------------------------
%	Common Tools
%----------------------------------------------------------------------------------------
%\begin{framed}
%\begin{lstlisting}[language=matlab]
%\end{lstlisting}
%\end{framed}

% \begin{bmatrix}
%0.5 & 0.6 \\ 
%0.7 & 0.8
%\end{bmatrix}

%\begin{figure}[h!]
%\centering
%\includegraphics[width=0.6\linewidth]{q10a.png}
%\label{fig:q10a}
%\end{figure}\\

%\begin{figure*}[!ht]
%\begin{subfigure}{.5\textwidth}
% \centering
%  \includegraphics[width=.5\linewidth]{p4_1.JPG}
%  \caption{Full set}
%  \label{fig:sfig1}
%\end{subfigure}
%\begin{subfigure}{.5\textwidth}
% \centering
%  \includegraphics[width=.5\linewidth]{P4_2.JPG}
%  \caption{Two each}
%  \label{fig:sfig2}
%\end{subfigure}%
%\caption{Part4 (a)}
%\label{fig:p4a}
%\end{figure*}

%\sum_{n=1}^{\infty} 2^{-n} = 1
%\prod_{i=a}^{b} f(i)

%\begin{equation}
%\begin{split}
%1+2+3+4+8x+7 & =1+2+3+4+4x+35 \\
%& \Rightarrow x=7
%\end{split}
%\end{equation}
%----------------------------------------------------------------------------------------
%	PROBLEM 1
%----------------------------------------------------------------------------------------

% To have just one problem per page, simply put a \clearpage after each problem
\begin{homeworkProblem}
\noindent \textit{VFE/ELBO}\\

(a)\\
\begin{equation}
\begin{split}
F(q) & = E_q[logp(\mathbf x|\mathbf z)] - D_{KL}(q(\mathbf z)||p(\mathbf z))\\
&= E_q[logp(\mathbf x|\mathbf z) - logq(\mathbf z) + logp(\mathbf z)]\\
&= E_q[log(p(\mathbf x|\mathbf z) p(\mathbf z) ) - logq(\mathbf z)]\\
&= E_q[log(p(\mathbf z|\mathbf x) p(\mathbf x) ) - logq(\mathbf z)]\ \#Bayes'\ Rule\\
&= E_q[logp(\mathbf x) + logp(\mathbf z|\mathbf x) - logq(\mathbf z)]\\
&=logp(\mathbf x) - E_q[logq(\mathbf z) - logp(\mathbf z|\mathbf x) ]\\
&= logp(\mathbf x) - D_{KL}(q(\mathbf z)||p(\mathbf z|\mathbf x)) 
\end{split}
\end{equation}

(b)\\
\begin{equation}
\begin{split}
D_{KL}(q(\mathbf z)||p(\mathbf z)) 
&= E_q[logq(\mathbf z) - logp(\mathbf z)]\\
&= E_q[log(\prod_{i=1}^{D}q_i(z_i)) - log(\prod_{i=1}^{D}p_i(z_i))]\\
&= E_q[\sum_{i=1}^{D}log(q_i(z_i)) - \sum_{i=1}^{D}log(p_i(z_i))]\\
&= E_q[\sum_{i=1}^{D}(logq_i(z_i) - logp_i(z_i))]\\
&= \sum_{i=1}^{D} E_q[logq_i(z_i) - logp_i(z_i)]\\
&= \sum_{i=1}^{D}D_{KL}(q_i(z_i)||p_i(z_i)) 
\end{split}
\end{equation}
\clearpage
(c)\\
\begin{equation}
\begin{split}
D_{KL}(q(z)||p(z)) 
&= E_q[logq(z) - logp(z)]\\
&= \int q(z)(logq(z)-logp(z)) dz\\
&= \int q(z)log\frac{q(z)}{p(z)} dz\\
&= \int q(z)log\frac{\frac{1}{\sqrt{2\pi}\sigma}exp(-\frac{(z-\mu)^2}{2\sigma^2})}{\frac{1}{\sqrt{2\pi}}exp(-\frac{z^2}{2})} dz\\
&= \int q(z)log\frac{\frac{1}{\sqrt{2\pi}\sigma}}{\frac{1}{\sqrt{2\pi}}}dz + \int q(z)log\frac{exp(-\frac{(z-\mu)^2}{2\sigma^2})}{exp(-\frac{z^2}{2})} dz\\
&= log\frac{1}{\sigma}\int q(z) dz + \int q(z)(-\frac{(z-\mu)^2}{2\sigma^2} + \frac{z^2}{2}) dz\\
&= log\frac{1}{\sigma} - \int q(z)\frac{(z-\mu)^2}{2\sigma^2}dz + \int q(z)\frac{z^2}{2} dz\\
&= log\frac{1}{\sigma} - \frac{1}{2\sigma^2}\int q(z)(z-\mu)^2dz + \frac{1}{2}\int q(z)z^2 dz\\
&= log\frac{1}{\sigma} - \frac{1}{2\sigma^2}Var(z) + \frac{1}{2}E_q[z^2]\\
&= log\frac{1}{\sigma} - \frac{1}{2}+ \frac{1}{2}(Var(z) + E_q[z]^2)\\
&= -\frac{1}{2}log(\sigma^2) -  \frac{1}{2} + \frac{1}{2}(\sigma^2+ \mu^2)\\
&= \frac{1}{2}(\sigma^2 + \mu^2 - log(\sigma^2) - 1)
\end{split}
\end{equation}
\clearpage
(d)\\
As defined in the question, to find $D_{KL}$, first, we want to find $\partial t/\partial \theta = \bar \theta = [\bar \mu, \bar \sigma]$
\begin{equation}
\begin{split}
\bar t &= 1\\
\bar r &= 1\\
\bar s &= -1\\
\bar z &= \bar r \frac{\partial r}{\partial z} + \bar s \frac{\partial s}{\partial z}\\
	&= 1\frac{\partial}{\partial z} logq(z) + (-1)\frac{\partial}{\partial z}logp(z)\\
	&= \frac{\partial}{\partial z} logq(z) - \frac{\partial}{\partial z}logp(z)\\
\frac{\partial}{\partial z} logq(z) &= \frac{\partial}{\partial z}log(\frac{1}{\sqrt{2\pi}\sigma}exp(-\frac{(z-\mu)^2}{2\sigma^2}))\\
	&=  \frac{\partial}{\partial z}(log(\frac{1}{\sqrt{2\pi}\sigma}) -\frac{(z-\mu)^2}{2\sigma^2})\\
	&= \frac{\partial}{\partial z}(-\frac{(z-\mu)^2}{2\sigma^2})\\
	&= -\frac{(z-\mu)}{\sigma^2}\\
\frac{\partial}{\partial z}logp(z) &= \frac{\partial}{\partial z}log(\frac{1}{\sqrt{2\pi}}exp(-\frac{z^2}{2}))\\
	&= \frac{\partial}{\partial z}(-\frac{z^2}{2})\\
	&= -z\\
\bar z &= -\frac{(z-\mu)}{\sigma^2} + z\\
\bar \mu &= \bar z = -\frac{(z-\mu)}{\sigma^2} + z\\
\bar \sigma &= \bar z \epsilon  = \epsilon(-\frac{(z-\mu)}{\sigma^2} + z)
\end{split}
\end{equation}

Since $\nabla_\theta D_{KL}(q(z)||p(z)) = E_\epsilon[\nabla_\theta t]$ where $\epsilon \sim N(0,1) $:
\begin{equation}
\begin{split}
\frac{\partial}{\partial \mu} D_{KL}(q(z)||p(z)) &= E_\epsilon[\bar \mu]\\
	&= E_\epsilon[-\frac{(z-\mu)}{\sigma^2} + z] \\
	&= E_\epsilon[-\frac{(\mu+\sigma\epsilon-\mu)}{\sigma^2} + \mu+\sigma\epsilon]\\
	&= E_\epsilon[-\frac{(\sigma\epsilon)}{\sigma^2}] + E_\epsilon[\mu] + E_\epsilon[\sigma\epsilon]\\
	&= \mu\ \ \ \# since\ \mu_\epsilon = 0
\end{split}
\end{equation}
\begin{equation}
\begin{split}
\frac{\partial}{\partial \sigma} D_{KL}(q(z)||p(z)) &= E_\epsilon[\bar \sigma]\\
	&= E_\epsilon[ \epsilon(-\frac{(z-\mu)}{\sigma^2} + z)]\\
	&= E_\epsilon[ \epsilon(-\frac{(\mu+\sigma\epsilon-\mu)}{\sigma^2} + \mu+\sigma\epsilon)]\\
	&= E_\epsilon[ \epsilon(-\frac{\epsilon}{\sigma} + \mu+\sigma\epsilon)]\\
	&= E_\epsilon[-\frac{\epsilon^2}{\sigma}] + E_\epsilon[\epsilon \mu] + E_\epsilon[\sigma\epsilon^2]\\
	&= -\frac{1}{\sigma}E_\epsilon[\epsilon^2] +  \mu E_\epsilon[\epsilon] + \sigma E_\epsilon[\epsilon^2]\\
    &= -\frac{1}{\sigma}(Var[\epsilon] + E_\epsilon[\epsilon]^2) +  \mu * 0 + \sigma (Var[\epsilon] + E_\epsilon[\epsilon]^2)\\
    &= -\frac{1}{\sigma}(1 + 0^2) +  \mu * 0 + \sigma (1 + 0^2)\\
    &= \sigma -\frac{1}{\sigma}
\end{split}
\end{equation}
\clearpage
\end{homeworkProblem}
%----------------------------------------------------------------------------------------
\end{document}
