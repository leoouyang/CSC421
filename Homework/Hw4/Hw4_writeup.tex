%%%%%%%%%%%%%%%%%%%%%%%%%%%%%%%%%%%%%%%%%
% Programming/Coding Assignment
% LaTeX Template
%
% This template has been downloaded from:
% http://www.latextemplates.com
%
% Original author:
% Ted Pavlic (http://www.tedpavlic.com)
%
% Note:
% The \lipsum[#] commands throughout this template generate dummy text
% to fill the template out. These commands should all be removed when 
% writing assignment content.
%
% This template uses a Perl script as an example snippet of code, most other
% languages are also usable. Configure them in the "CODE INCLUSION 
% CONFIGURATION" section.
%
%%%%%%%%%%%%%%%%%%%%%%%%%%%%%%%%%%%%%%%%%

%----------------------------------------------------------------------------------------
%	PACKAGES AND OTHER DOCUMENT CONFIGURATIONS
%----------------------------------------------------------------------------------------

%\documentclass{article}
\documentclass[11pt]{article}
\usepackage{fancyhdr} % Required for custom headers
\usepackage{lastpage} % Required to determine the last page for the footer
\usepackage{extramarks} % Required for headers and footers
\usepackage[usenames,dvipsnames]{color} % Required for custom colors
\usepackage{graphicx} % Required to insert images
\usepackage{subcaption}
\usepackage{listings} % Required for insertion of code
\usepackage{courier} % Required for the courier font
\usepackage{amsmath}
\usepackage{framed}

% Margins
\topmargin=-0.45in
\evensidemargin=0in
\oddsidemargin=0in
\textwidth=6.5in
\textheight=9.0in
\headsep=0.25in

\linespread{1.1} % Line spacing

% Set up the header and footer
\pagestyle{fancy}
\lhead{\hmwkAuthorName} % Top left header
\chead{\hmwkClass\ (\hmwkClassTime): \hmwkTitle} % Top center head
%\rhead{\firstxmark} % Top right header
\lfoot{\lastxmark} % Bottom left footer
\cfoot{} % Bottom center footer
\rfoot{Page\ \thepage\ of\ \protect\pageref{LastPage}} % Bottom right footer
\renewcommand\headrulewidth{0.4pt} % Size of the header rule
\renewcommand\footrulewidth{0.4pt} % Size of the footer rule

\setlength\parindent{0pt} % Removes all indentation from paragraphs

%----------------------------------------------------------------------------------------
%	CODE INCLUSION CONFIGURATION
%----------------------------------------------------------------------------------------

\definecolor{mygreen}{rgb}{0,0.6,0}
\definecolor{mygray}{rgb}{0.5,0.5,0.5}
\definecolor{mymauve}{rgb}{0.58,0,0.82}

\lstset{ %
  backgroundcolor=\color{white},   % choose the background color
  basicstyle=\footnotesize,        % size of fonts used for the code
  breaklines=true,                 % automatic line breaking only at whitespace
  captionpos=b,                    % sets the caption-position to bottom
  commentstyle=\color{mygreen},    % comment style
  escapeinside={\%*}{*)},          % if you want to add LaTeX within your code
  keywordstyle=\color{blue},       % keyword style
  stringstyle=\color{mymauve},     % string literal style
}

%----------------------------------------------------------------------------------------
%	DOCUMENT STRUCTURE COMMANDS
%	Skip this unless you know what you're doing
%----------------------------------------------------------------------------------------

% Header and footer for when a page split occurs within a problem environment
\newcommand{\enterProblemHeader}[1]{
%\nobreak\extramarks{#1}{#1 continued on next page\ldots}\nobreak
%\nobreak\extramarks{#1 (continued)}{#1 continued on next page\ldots}\nobreak
}

% Header and footer for when a page split occurs between problem environments
\newcommand{\exitProblemHeader}[1]{
%\nobreak\extramarks{#1 (continued)}{#1 continued on next page\ldots}\nobreak
%\nobreak\extramarks{#1}{}\nobreak
}

\setcounter{secnumdepth}{0} % Removes default section numbers
\newcounter{homeworkProblemCounter} % Creates a counter to keep track of the number of problems
\setcounter{homeworkProblemCounter}{0}

\newcommand{\homeworkProblemName}{}
\newenvironment{homeworkProblem}[1][Problem \arabic{homeworkProblemCounter}]{ % Makes a new environment called homeworkProblem which takes 1 argument (custom name) but the default is "Problem #"
\stepcounter{homeworkProblemCounter} % Increase counter for number of problems
\renewcommand{\homeworkProblemName}{#1} % Assign \homeworkProblemName the name of the problem
\section{\homeworkProblemName} % Make a section in the document with the custom problem count
\enterProblemHeader{\homeworkProblemName} % Header and footer within the environment
}{
\exitProblemHeader{\homeworkProblemName} % Header and footer after the environment
}

\newcommand{\problemAnswer}[1]{ % Defines the problem answer command with the content as the only argument
\noindent\framebox[\columnwidth][c]{\begin{minipage}{0.98\columnwidth}#1\end{minipage}} % Makes the box around the problem answer and puts the content inside
}

\newcommand{\homeworkSectionName}{}
\newenvironment{homeworkSection}[1]{ % New environment for sections within homework problems, takes 1 argument - the name of the section
\renewcommand{\homeworkSectionName}{#1} % Assign \homeworkSectionName to the name of the section from the environment argument
\subsection{\homeworkSectionName} % Make a subsection with the custom name of the subsection
\enterProblemHeader{\homeworkProblemName\ [\homeworkSectionName]} % Header and footer within the environment
}{
\enterProblemHeader{\homeworkProblemName} % Header and footer after the environment
}

%----------------------------------------------------------------------------------------
%	NAME AND CLASS SECTION
%----------------------------------------------------------------------------------------

\newcommand{\hmwkTitle}{Written Homework 3} % Assignment title
\newcommand{\hmwkDueDate}{Thrusday, Mar 7, 2019} % Due date
\newcommand{\hmwkClass}{CSC421} % Course/class
\newcommand{\hmwkClassTime}{LEC 5101} % Class/lecture time
\newcommand{\hmwkAuthorName}{Zhongtian Ouyang} % Your name
\newcommand{\hmwkAuthorID}{1002341012} % Your ID


%----------------------------------------------------------------------------------------
%	TITLE PAGE
%----------------------------------------------------------------------------------------

\title{
\vspace{2in}
\textmd{\textbf{\hmwkClass:\ \hmwkTitle}}\\
\normalsize\vspace{0.1in}\small{Due\ on\ \hmwkDueDate}\\
\vspace{0.1in}
\vspace{3in}
}

\author{\textbf{\hmwkAuthorName}\\ \textbf{\hmwkAuthorID}}

\date{} % Insert date here if you want it to appear below your name

%----------------------------------------------------------------------------------------\
\begin{document}

\maketitle
\clearpage

%----------------------------------------------------------------------------------------
%	Common Tools
%----------------------------------------------------------------------------------------
%\begin{framed}
%\begin{lstlisting}[language=matlab]
%\end{lstlisting}
%\end{framed}

% \begin{bmatrix}
%0.5 & 0.6 \\ 
%0.7 & 0.8
%\end{bmatrix}

%\begin{figure}[h!]
%\centering
%\includegraphics[width=0.6\linewidth]{q10a.png}
%\label{fig:q10a}
%\end{figure}\\

%\begin{figure*}[!ht]
%\begin{subfigure}{.5\textwidth}
% \centering
%  \includegraphics[width=.5\linewidth]{p4_1.JPG}
%  \caption{Full set}
%  \label{fig:sfig1}
%\end{subfigure}
%\begin{subfigure}{.5\textwidth}
% \centering
%  \includegraphics[width=.5\linewidth]{P4_2.JPG}
%  \caption{Two each}
%  \label{fig:sfig2}
%\end{subfigure}%
%\caption{Part4 (a)}
%\label{fig:p4a}
%\end{figure*}

%\sum_{n=1}^{\infty} 2^{-n} = 1
%\prod_{i=a}^{b} f(i)

%\begin{equation}
%\begin{split}
%1+2+3+4+8x+7 & =1+2+3+4+4x+35 \\
%& \Rightarrow x=7
%\end{split}
%\end{equation}
%----------------------------------------------------------------------------------------
%	PROBLEM 1
%----------------------------------------------------------------------------------------

% To have just one problem per page, simply put a \clearpage after each problem
\begin{homeworkProblem}
\noindent \textit{LSTM Gradient}\\
(a)
\begin{equation}
\begin{split}
\overline{h^{(t)}} &= \overline{i^{(t+1)}} i^{(t+1)} (1-i^{(t+1)}) w_{ih}\\
							&+ \overline{f^{(t+1)}} f^{(t+1)} (1-f^{(t+1)}) w_{fh}\\
							&+ \overline{o^{(t+1)}} o^{(t+1)} (1-o^{(t+1)}) w_{oh}\\
							&+ \overline{g^{(t+1)}} (1-{g^{(t+1)}}^2) w_{gh}\\
\overline{c^{(t)}} &= \overline{c^{(t+1)}}f^{(t+1)} + \overline{h^{(t)}}o^{(t)}(1- (tanh(c^{(t)}))^2)\\
\overline{g^{(t)}} &= \overline{c^{(t)}}i^{(t)}\\
\overline{o^{(t)}} &= \overline{h^{(t)}}tanh(c^{(t)})\\
\overline{f^{(t)}} &= \overline{c^{(t)}}c^{(t-1)}\\
\overline{i^{(t)}} &= \overline{c^{(t)}}g^{(t)}\\
\end{split}
\end{equation}
If $h^{(t)}$ is used in $y^{(t)}$, $\overline{h^{(t)}} += \overline{y^{(t)}}(\partial y^{(t)}/ \partial h^{(t)})$\\
If $h^{(t)}$ is used in Loss function $L$, $\overline{h^{(t)}} += \overline{L}(\partial L/ \partial h^{(t)})$\\

(b)\\
$$\overline{w_{ix}} = \sum_t \overline{i^{(t)}}i^{(t)}(1-i^{(t)})x^{(t)}$$

\end{homeworkProblem}
\clearpage
%----------------------------------------------------------------------------------------
%	PROBLEM 2
%----------------------------------------------------------------------------------------

\begin{homeworkProblem}
\noindent \textit{Multidimensional RNN}\\
(a)\\
Number of Weights:\\
$W_{in}^T: D \times H$\\
$W_{W}^T: H\times H$\\
$W_{W}^T: H\times H$\\
Total: $(D+2H) \times H$\\

Number of arithmetic operations for an $h^{(i,j)}$\\
Activation function elementwise(assume $n_a$ arithmatic operations to evaluate $\phi(x)$): $n_aH$\\
Addition of vectors inside activation function: 2H\\
$W_{in}^Tx^{(i,j)}: H \times (D\ multiplications + D-1\ addtions) = H \times (2D-1)$\\
$W_{W}^T h^{(i-1,j)}: H \times (H\ multiplications + H-1\ addtions) = H \times (2H-1)$\\
$W_{N}^T h^{(i,j-1)}: H \times (H\ multiplications + H-1\ addtions) = H \times (2H-1)$\\
Total:$H\times(n_a + 2 + 2D-1 + 2H-1 + 2H-1) = H\times(n_a + 2D + 4H - 1) = O(H \times (D + H))$\\
For $G \times G$ hidden vectors in the grid: $O(G \times G \times H \times (D + H)) = O(G^2  \times H \times (D + H))$\\

(b)\\
Assume that addition and activation function can be done in the same step as matrix-vector multiplications.\\
$2G - 1$ steps will be needed to computed the hidden activations of the $G \times G$ grid. \\
One way of doing is computing the activations in the following sequence:\\
Step 1: $h^{(0,0)}$\\
Step 2: $h^{(1,0)}$, $h^{(0,1)}$\\
Step 3: $h^{(0,2)}$, $h^{(1,1)}$, $h^{(2,0)}$\\
...\\
Step $2G - 2$: $h^{(G-1,G)}$, $h^{(G, G-1)}$\\
Step $2G - 1$: $h^{(G,G)}$\\
In this sequence, when we do a step, all the information for calculating each hidden activations is.\\

(c)\\
Disadvantage: The calculations for a conv net can be well parallelized. Less seqential steps are required to compute an conv net with same dimension compare to an MDRNN.\\
Advantage: MDRNN can capture the sequential relationship between the datas and extract more information compare to CNN.\\
\end{homeworkProblem}
\clearpage
%----------------------------------------------------------------------------------------
%	PROBLEM 2
%----------------------------------------------------------------------------------------
\begin{homeworkProblem}
\noindent \textit{Reversibility}\\

(a)
\begin{equation}
\begin{split}
\mathbf s^{(k+1)} &= (\boldsymbol \theta^{(k+1)}, \mathbf p^{(k+1)})\\
\boldsymbol \theta^{(k)} &= \boldsymbol \theta^{(k+1)} - \mathbf p^{(k+1)}\\
\mathbf p^{(k)} &= \frac{\mathbf p^{(k+1)} + \alpha\nabla J(\boldsymbol \theta^{(k)})}{\beta}\\
\mathbf s^{(k)} &= (\boldsymbol \theta^{(k)}, \mathbf p^{(k)})
\end{split}
\end{equation}

(b)\\
\begin{equation}
\begin{split}
\frac{\partial \mathbf s^{(k+1)}}{\partial \mathbf s^{(k)}} 
\end{split}
\end{equation}
Since the top half and bottom half of the matrix is identical, the rank is D < 2D. The determinant of the matrix is 0\\
\end{homeworkProblem}
\clearpage
%----------------------------------------------------------------------------------------

\end{document}
