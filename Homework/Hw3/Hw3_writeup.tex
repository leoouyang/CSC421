%%%%%%%%%%%%%%%%%%%%%%%%%%%%%%%%%%%%%%%%%
% Programming/Coding Assignment
% LaTeX Template
%
% This template has been downloaded from:
% http://www.latextemplates.com
%
% Original author:
% Ted Pavlic (http://www.tedpavlic.com)
%
% Note:
% The \lipsum[#] commands throughout this template generate dummy text
% to fill the template out. These commands should all be removed when 
% writing assignment content.
%
% This template uses a Perl script as an example snippet of code, most other
% languages are also usable. Configure them in the "CODE INCLUSION 
% CONFIGURATION" section.
%
%%%%%%%%%%%%%%%%%%%%%%%%%%%%%%%%%%%%%%%%%

%----------------------------------------------------------------------------------------
%	PACKAGES AND OTHER DOCUMENT CONFIGURATIONS
%----------------------------------------------------------------------------------------

%\documentclass{article}
\documentclass[11pt]{article}
\usepackage{fancyhdr} % Required for custom headers
\usepackage{lastpage} % Required to determine the last page for the footer
\usepackage{extramarks} % Required for headers and footers
\usepackage[usenames,dvipsnames]{color} % Required for custom colors
\usepackage{graphicx} % Required to insert images
\usepackage{subcaption}
\usepackage{listings} % Required for insertion of code
\usepackage{courier} % Required for the courier font
\usepackage{amsmath}
\usepackage{framed}

% Margins
\topmargin=-0.45in
\evensidemargin=0in
\oddsidemargin=0in
\textwidth=6.5in
\textheight=9.0in
\headsep=0.25in

\linespread{1.1} % Line spacing

% Set up the header and footer
\pagestyle{fancy}
\lhead{\hmwkAuthorName} % Top left header
\chead{\hmwkClass\ (\hmwkClassTime): \hmwkTitle} % Top center head
%\rhead{\firstxmark} % Top right header
\lfoot{\lastxmark} % Bottom left footer
\cfoot{} % Bottom center footer
\rfoot{Page\ \thepage\ of\ \protect\pageref{LastPage}} % Bottom right footer
\renewcommand\headrulewidth{0.4pt} % Size of the header rule
\renewcommand\footrulewidth{0.4pt} % Size of the footer rule

\setlength\parindent{0pt} % Removes all indentation from paragraphs

%----------------------------------------------------------------------------------------
%	CODE INCLUSION CONFIGURATION
%----------------------------------------------------------------------------------------

\definecolor{mygreen}{rgb}{0,0.6,0}
\definecolor{mygray}{rgb}{0.5,0.5,0.5}
\definecolor{mymauve}{rgb}{0.58,0,0.82}

\lstset{ %
  backgroundcolor=\color{white},   % choose the background color
  basicstyle=\footnotesize,        % size of fonts used for the code
  breaklines=true,                 % automatic line breaking only at whitespace
  captionpos=b,                    % sets the caption-position to bottom
  commentstyle=\color{mygreen},    % comment style
  escapeinside={\%*}{*)},          % if you want to add LaTeX within your code
  keywordstyle=\color{blue},       % keyword style
  stringstyle=\color{mymauve},     % string literal style
}

%----------------------------------------------------------------------------------------
%	DOCUMENT STRUCTURE COMMANDS
%	Skip this unless you know what you're doing
%----------------------------------------------------------------------------------------

% Header and footer for when a page split occurs within a problem environment
\newcommand{\enterProblemHeader}[1]{
%\nobreak\extramarks{#1}{#1 continued on next page\ldots}\nobreak
%\nobreak\extramarks{#1 (continued)}{#1 continued on next page\ldots}\nobreak
}

% Header and footer for when a page split occurs between problem environments
\newcommand{\exitProblemHeader}[1]{
%\nobreak\extramarks{#1 (continued)}{#1 continued on next page\ldots}\nobreak
%\nobreak\extramarks{#1}{}\nobreak
}

\setcounter{secnumdepth}{0} % Removes default section numbers
\newcounter{homeworkProblemCounter} % Creates a counter to keep track of the number of problems
\setcounter{homeworkProblemCounter}{0}

\newcommand{\homeworkProblemName}{}
\newenvironment{homeworkProblem}[1][Problem \arabic{homeworkProblemCounter}]{ % Makes a new environment called homeworkProblem which takes 1 argument (custom name) but the default is "Problem #"
\stepcounter{homeworkProblemCounter} % Increase counter for number of problems
\renewcommand{\homeworkProblemName}{#1} % Assign \homeworkProblemName the name of the problem
\section{\homeworkProblemName} % Make a section in the document with the custom problem count
\enterProblemHeader{\homeworkProblemName} % Header and footer within the environment
}{
\exitProblemHeader{\homeworkProblemName} % Header and footer after the environment
}

\newcommand{\problemAnswer}[1]{ % Defines the problem answer command with the content as the only argument
\noindent\framebox[\columnwidth][c]{\begin{minipage}{0.98\columnwidth}#1\end{minipage}} % Makes the box around the problem answer and puts the content inside
}

\newcommand{\homeworkSectionName}{}
\newenvironment{homeworkSection}[1]{ % New environment for sections within homework problems, takes 1 argument - the name of the section
\renewcommand{\homeworkSectionName}{#1} % Assign \homeworkSectionName to the name of the section from the environment argument
\subsection{\homeworkSectionName} % Make a subsection with the custom name of the subsection
\enterProblemHeader{\homeworkProblemName\ [\homeworkSectionName]} % Header and footer within the environment
}{
\enterProblemHeader{\homeworkProblemName} % Header and footer after the environment
}

%----------------------------------------------------------------------------------------
%	NAME AND CLASS SECTION
%----------------------------------------------------------------------------------------

\newcommand{\hmwkTitle}{Written Homework 3} % Assignment title
\newcommand{\hmwkDueDate}{Thrusday, Mar 7, 2019} % Due date
\newcommand{\hmwkClass}{CSC421} % Course/class
\newcommand{\hmwkClassTime}{LEC 5101} % Class/lecture time
\newcommand{\hmwkAuthorName}{Zhongtian Ouyang} % Your name
\newcommand{\hmwkAuthorID}{1002341012} % Your ID


%----------------------------------------------------------------------------------------
%	TITLE PAGE
%----------------------------------------------------------------------------------------

\title{
\vspace{2in}
\textmd{\textbf{\hmwkClass:\ \hmwkTitle}}\\
\normalsize\vspace{0.1in}\small{Due\ on\ \hmwkDueDate}\\
\vspace{0.1in}
\vspace{3in}
}

\author{\textbf{\hmwkAuthorName}\\ \textbf{\hmwkAuthorID}}

\date{} % Insert date here if you want it to appear below your name

%----------------------------------------------------------------------------------------\
\begin{document}

\maketitle
\clearpage

%----------------------------------------------------------------------------------------
%	Common Tools
%----------------------------------------------------------------------------------------
%\begin{framed}
%\begin{lstlisting}[language=matlab]
%\end{lstlisting}
%\end{framed}

% \begin{bmatrix}
%0.5 & 0.6 \\ 
%0.7 & 0.8
%\end{bmatrix}

%\begin{figure}[h!]
%\centering
%\includegraphics[width=0.6\linewidth]{q10a.png}
%\label{fig:q10a}
%\end{figure}\\

%\begin{figure*}[!ht]
%\begin{subfigure}{.5\textwidth}
% \centering
%  \includegraphics[width=.5\linewidth]{p4_1.JPG}
%  \caption{Full set}
%  \label{fig:sfig1}
%\end{subfigure}
%\begin{subfigure}{.5\textwidth}
% \centering
%  \includegraphics[width=.5\linewidth]{P4_2.JPG}
%  \caption{Two each}
%  \label{fig:sfig2}
%\end{subfigure}%
%\caption{Part4 (a)}
%\label{fig:p4a}
%\end{figure*}

%\sum_{n=1}^{\infty} 2^{-n} = 1
%\prod_{i=a}^{b} f(i)

%\begin{equation}
%\begin{split}
%1+2+3+4+8x+7 & =1+2+3+4+4x+35 \\
%& \Rightarrow x=7
%\end{split}
%\end{equation}
%----------------------------------------------------------------------------------------
%	PROBLEM 1
%----------------------------------------------------------------------------------------

% To have just one problem per page, simply put a \clearpage after each problem
\begin{homeworkProblem}
\noindent \textit{Dropout}\\

(a)
\begin{equation}
\begin{split}
E(y) &= E[\sum_{j} m_j w_j x_j] \\
&= E[m_1w_1x_1 + m_2w_2x_2 + ... + m_Jw_Jx_J] \\
&= E[m_1w_1x_1] + ... + E[m_Jw_Jx_J]\\
&= w_1x_1E[m_1] + ... + w_Jx_JE[m_J]\\
&= w_1x_1(\frac{1}{2}) + ... + m_Jw_J(\frac{1}{2}) = \frac{1}{2}\sum_j w_jx_j = \frac{1}{2} \mathbf{w}\cdot \mathbf{x}
\end{split}
\end{equation}
\begin{equation}
\begin{split}
Var(y) &= Var(\sum_{j} m_j w_j x_j)\\
&= \sum_{j}Var(m_j w_j x_j),\ since\ they\ are\ independent \\
&= \sum_{j}w_j^2 x_j^2 Var(m_j)\\
&= \sum_{j}w_j^2 x_j^2 (E[m_j^2] - E[m_j]^2)\\
&= \sum_{j}w_j^2 x_j^2 (\frac{1}{2} - \frac{1}{4}),\ E[m_j^2]= E[m_j] \ because\ m_j\ is\ either\ 0\ or\ 1,\ m_j = m_j^2\\
&= \frac{1}{4}\sum_{j}w_j^2 x_j^2 = \frac{1}{4}(\mathbf{w} \cdot \mathbf{w})\cdot (\mathbf{x}\cdot \mathbf{x})
\end{split}
\end{equation}

(b)\\
Since $E(y) = \tilde y= \frac{1}{2}\sum_j w_j x_j = \sum_j (\frac{1}{2}w_j) x_j$, $\tilde{w_j} = \frac{1}{2}w_j$.\\
\clearpage

(c)\\
\begin{equation}
\begin{split}
J &= \frac{1}{2N}\sum_{i=1}^{N} E[(y^{(i)} - t^{(i)})^2]\\
&= \frac{1}{2N}\sum_{i=1}^{N} E[{y^{(i)}}^2 - 2y^{(i)}t^{(i)} + {t^{(i)}}^2]\\
&=  \frac{1}{2N}\sum_{i=1}^{N} (E[{y^{(i)}}^2] - E[2y^{(i)}t^{(i)}] +E[{t^{(i)}}^2])\\
&= \frac{1}{2N}\sum_{i=1}^{N} ((Var[y^{(i)}] + E[y^{(i)}]^2) - 2t^{(i)}E[y^{(i)}] + {t^{(i)}}^2)\\
&=\frac{1}{2N}\sum_{i=1}^{N} ((\frac{1}{4}\sum_{j}w_j^2 x_j^2 + (\frac{1}{2}\sum_j w_jx^{(i)}_j)^2) - 2t^{(i)}\frac{1}{2}\sum_j w_jx^{(i)}_j + {t^{(i)}}^2)\\
&= \frac{1}{2N}\sum_{i=1}^{N} ((\frac{1}{4}\sum_{j}(2\tilde{w_j})^2 x_j^2 + (\frac{1}{2}\sum_j 2\tilde{w_j} x^{(i)}_j)^2) - 2t^{(i)}\frac{1}{2}\sum_j 2\tilde{w_j} x^{(i)}_j + {t^{(i)}}^2)\\
&= \frac{1}{2N}\sum_{i=1}^{N} (\sum_{j}\tilde{w_j}^2 x_j^2 + (\sum_j \tilde{w_j} x^{(i)}_j)^2 - 2t^{(i)}\sum_j \tilde{w_j} x^{(i)}_j + {t^{(i)}}^2)\\
&= \frac{1}{2N}\sum_{i=1}^{N} (\sum_{j}\tilde{w_j}^2 x_j^2 + ({\tilde y}^{(i)})^2 - 2t^{(i)}\tilde y^{(i)} + {t^{(i)}}^2)\\
&= \frac{1}{2N}\sum_{i=1}^{N}(({\tilde y}^{(i)})^2 - 2t^{(i)}\tilde y^{(i)} + {t^{(i)}}^2) + \frac{1}{2N}\sum_{i=1}^{N} \sum_{j}\tilde{w_j}^2 x_j^2\\
&=\frac{1}{2N}\sum_{i=1}^{N}(\tilde y^{(i)} - t^{(i)})^2 + R(\tilde w_1, ..., \tilde w_D)
\end{split}
\end{equation}
\end{homeworkProblem}
\clearpage
%----------------------------------------------------------------------------------------
%	PROBLEM 2
%----------------------------------------------------------------------------------------

\begin{homeworkProblem}
\noindent \textit{RNN}\\
$$
U = 
 \begin{bmatrix}
1 & 1\\ 
1 & 1\\ 
1 & 1 
\end{bmatrix}
$$

$$
V = 
\begin{bmatrix}
1 & -1 & 2
\end{bmatrix}
$$

$$
W = 
 \begin{bmatrix}
h_1^{(t)} \rightarrow h_1^{(t+1)} & h_2^{(t)} \rightarrow h_1^{(t+1)} & h_3^{(t)} \rightarrow h_1^{(t+1)}\\ 
h_1^{(t)} \rightarrow h_2^{(t+1)} & h_2^{(t)} \rightarrow h_2^{(t+1)} & h_3^{(t)} \rightarrow h_2^{(t+1)}\\ 
h_1^{(t)} \rightarrow h_3^{(t+1)} & h_2^{(t)} \rightarrow h_3^{(t+1)} & h_3^{(t)} \rightarrow h_3^{(t+1)} 
\end{bmatrix}
=
 \begin{bmatrix}
0& 1 & 0\\ 
0& 1 & 0\\ 
0& 1 & 0 
\end{bmatrix}
$$

$$
b_h = 
 \begin{bmatrix}
-0.5 \\ -1.5 \\ -2.5 \\ 
\end{bmatrix}
$$

$$
b_y = -0.5
$$
\end{homeworkProblem}
\clearpage
%----------------------------------------------------------------------------------------

\end{document}
